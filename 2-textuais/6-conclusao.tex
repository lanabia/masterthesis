\chapter{Conclusion}
\label{chap:conclusion}

% O que me chamou a atenção na conclusão foi que vc coloca a metodologia como introduzida por vc e não foi, revisão sistemática com uso de grounded thoery não é contribuição sua. Suavize isso, por favor na conclusão e em toda a tese, se vc tiver feito algo nesse sentido. A sua contribuição está na aplicação delas para o seu tema e os resultados que encontrou.
% O meu ponto é que vc não pode considerar a metodologia como sua contribuição, ela já existe. O que vc fez foi aplicá-la em  um domínio específico, esta é sua contribuição secundária. Sua contribuição principal são os resultados da revisão sistemática e a análise dos dados com a grounded theory. Ficou mais claro?

The present chapter concludes the master's thesis, synthesizing the contributions obtained in this research, following the organization described next. Section \ref{sec:conclusion-summary} gives an overview of the research. Section \ref{sec:conclusion-mainresults} summarizes the main results of the research. Section \ref{sec:conclusion-contributions} presents the main contributions as also papers published. \ref{sec:conclusion-limitations} discusses the limitations of this work. Finally, Section \ref{sec:conclusion-future} motivates the development of future work.

\section{Overview}
\label{sec:conclusion-summary}

The applications for people who are blind have many needs due to the target audience and unique characteristics with multimodal interfaces, moreover, a lot of the applications for people who are blind or visually impaired aims to improve a cognitive skill, such as cognitive enhancement in O\&M, wayfinding, and navigation skills, and thus supporting the user in daily lives. 

It is important to point out that the use of evidence-based is essential to measure the real impact of the technologies on their software engineering process. Among so many \gls{EBSE} methods and their characteristics, the experiments to evaluate the cognitive impact of an application or a technology have faults in the experiment process.

As so, the main goal of this research was to provide an approach to support impact evaluations in technologies for cognitive development and enhancement of people who are blind or visually impaired. The second goal seeks to summarize the empirical evidence concerning the strengths and limitations of the experiments to this purpose using Systematic Literature Review and Grounded Theory \cite{Kitchenham2007}. As results, we produce a set of guidelines to support cognitive impact evaluation. These guidelines can be used in different contexts of society, such as special Institutes and Schools, research groups and practitioners. The guidelines provide to researchers and practitioners an appropriate guide to assess applications regarding the cognitive effectiveness, that is the main contribution of this research.

\section{Main results}
\label{sec:conclusion-mainresults}
The methodology used in this work consisted of two main methods: a Systematic Literature Review and a Grounded Theory process. The Systematic Literature Review, a secondary study, was conducted to observe how cognitive impact evaluation is addressed in such experiments. Based on the findings, principles of Grounded Theory were applied to analyze, organize and relate the concepts. After conducting a Systematic Literature Review and a Grounded Theory analysis of the results founded in the Systematic Literature Review, we conclude that it is necessary to have a more scientific basis regarding the methodology and rigor of the experiments. {It is understood that in addition to the responsibility of reviewers and member of the board in doctoral and master defenses, the researchers themselves should be aware of the positive and negative impact that their research may have on society.} In this way, the results obtained become less reliable and may discredit the use of the presented application or technology. 

This perception is based on data of experiments retrieved from the scientific papers founded in the Systematic Literature Review which report the experiment, and the empirical study cannot be distinguished from its reporting \cite{Wohlin2000}. Then, the papers with experiments are the primary source of information for judging the quality of the study.

The experiments that do not report well their characteristics, showed in Section \ref{sec:results-empirical}, become weak in one of the main points of the experiment: the repeatability. Without data such as variables and hypotheses explicit in the text, it is difficult to reproduce it under the same conditions. If the differences in factors and settings are well documented and analyzed, more knowledge may be gained from replicated studies \cite{Wohlin2000}. Besides, resource data, such as time, costs and labor, make it easier to reproduce later.

We understand the papers retrieved could describe the whole research and the experiment report is only a part of the paper. However, if the paper objective is not to report the experiment, it is essential to append a separated document as a technical report with all necessary information. According to \citeonline{Wohlin2000}, journal or conference articles are appropriate reports which have peer researchers as the primary audience; however he emphasizes the possibility of accompanying technical reports.

%Most of the authors of the papers retrieved are experts in the presented application or technology.

The importance of the experiment is to consider the formalism in terms of the cognition assumptions underpinning that knowledge. This concern introduces challenges in concepts and formalism of the area of cognition that must be well studied and understood within the experiment. Thus, it is important to involve other disciplines to test an in-depth hypothesis, so as to guarantee rigor in the result, as indicated by \citeonline{Kitchenham2002PreliminaryEngineering}. It states that without the link from theory to the hypothesis, empirical results cannot contribute to a broader body of knowledge.

\section{Contributions}
\label{sec:conclusion-contributions}

The main contribution of this work is to provide guidelines for evaluating the impact and effectiveness for cognitive development and enhancement in people who are blind or visually impaired, considering the main aspects of multimodal interfaces. Application designers and practitioners can use the set of guidelines to evaluate and improve their applications.

%This set includes the following guidelines.

%\begin{itemize}
%    \item G1 - Specify the Hypotheses;
%    \item G2 - Define and report the experiment variables;
%    \item G3 - Choose a valid sample;
%    \item G4 - Plan and Report Ethical Concepts; and
%    \item G5 - Specify the resources available to all parties involved.
%\end{itemize}

Also, with this work, we expected to have created a bibliographic review of the cognitive impact evaluation based on the steps of the systematic literature review approach. It organizes the state-of-the-art of this research area making the information summarized. Having the research findings organized in such a manner can stimulate research and lead to the extension of knowledge.

In Section \ref{chap:resultados}, we present an overview of the papers encountered, highlighting the principal authors and research groups. This overview is an analysis not much explored, but it shows a global summary of papers distribution among the institutions, countries, and authors. We see the phenomenon of publication and integration of institutions in the software engineering and human-computer interaction areas, that could be explored in future work.

In contribution, we offer the Systematic Literature Review spreadsheet\footnote{\url{https://www.dropbox.com/s/dvvn44cymqsqguy/Systematic\%20Review\%20-\%20Mesquita\%20L.\%202018.xlsm?dl=0}}, available online, for other researchers interested in expanding or replicating a review.

%\section{Publications}
%\label{sec:conclusion-publications}
Two papers were published in conferences, as a direct result of the research performed in this work. Table \ref{tab:Publications_as_a_direct_consequence_of_this_thesis_research} presents the references for these papers. The first paper \cite{Mesquita2017InBlind} is a proposal of this research with early results published in a Workshop of Thesis and Dissertation (WTD). The next paper \cite{Mesquita2018CognitiveReview} concerns the description of the results of the Systematic Literature Review, the first step of the methodology (Section \ref{chap:metodologia}).

During the development of this research, other four papers in the areas of HCI interaction and software engineering were published, not directly related to the contribution here presented, but somehow contributing to the acquired knowledge and research skills (Table \ref{tab:Secondary_publications_during_the_development_of_this_masters_thesis}).

\begin{table}[h]
	\captionsetup{width=16cm}%Deixe da mesma largura que a tabela
	\Caption{\label{tab:Publications_as_a_direct_consequence_of_this_thesis_research} Publications as a direct consequence of this thesis research}%
	\IBGEtab{}{%
		\begin{tabular}{m{14cm}m{1.2cm}}
			\toprule
		    Citation & Qualis \footnote{Based on classification of quadrennium 2013-2016 (\url{https://sucupira.capes.gov.br)}.} \\
			\midrule \midrule
			\smallskip
			Mesquita, L., Sánchez, J., and Andrade, R. M. C. (2018).\textbf{ Cognitive Impact Evaluation of Multimodal Interfaces for Blind People: Towards a Systematic Review.} Human-Computer Interaction International Conference (HCII). Las Vegas, USA. & B2 \smallskip\\ \hline
			\smallskip
			Mesquita, L., and Sánchez, J. (2017). \textbf{In Search of a Multimodal Interfaces Impact Evaluation Model for People Who Are Blind.} In Workshop of Theses and Dissertations (WTD) in the 16th Brazilian Symposium on Human Factors in Computing Systems (IHC 2017). Joinville, Brazil. & B2 \smallskip\\
			\bottomrule 
		\end{tabular}%
	}{%
	\Fonte{Produced by the author.}%
%	\Nota{esta é uma nota, que diz que os dados são baseados na	regressão linear.}%
%	\Nota[Anotações]{uma anotação adicional, seguida de várias outras.}%
    }
    \end{table}

\begin{table}[h]
	\captionsetup{width=16cm}%Deixe da mesma largura que a tabela
	\Caption{\label{tab:Secondary_publications_during_the_development_of_this_masters_thesis} Secondary publications during the development of this master’s thesis}%
	\IBGEtab{}{%
		\begin{tabular}{m{14cm}m{1cm}}
			\toprule
			Citation & Qualis \\
			\midrule \midrule
			\smallskip
		    Mesquita, L., Ismayle Sousa Santos, Bruno Aragão, Tales Nogueira, and Rossana Andrade. (2017). Modelagem Interativa de um Processo de Desenvolvimento com Base na Percepção da Equipe: Um Relato de Experiência. CEUR Workshop Proceedings, 2065, 54–57. & B2 \smallskip\\ \hline
		    \smallskip
		    Lucas, R., Almeida, A., Mesquita, L., Almeida, R. L. A., Mesquita, L. B., and Carvalho, R. M. (2016). Quando a Tecnologia apoia a Mobilidade Urbana : Uma Avaliação sobre a Experiência do Usuário com Aplicações Móveis Quando a Tecnologia apoia a Mobilidade Urbana : Uma Avaliação sobre a Experiência do Usuário com Aplicações Móveis, (October). & B2 \smallskip\\ \hline
		    \smallskip
		    Almeida, R. L. A., Mesquita, L. B., Carvalho, R. M., and Andrade, R. M. C. (2017). When technology supports urban mobility: Improvements for mobile applications based on a UX evaluation. Lecture Notes in Computer Science (including subseries Lecture Notes in Artificial Intelligence and Lecture Notes in Bioinformatics) (Vol. 10272 LNCS). & B2 \smallskip\\ \hline
		    \smallskip
		    Darin, T., Andrade, R., Macedo, J., Araújo, D., Mesquita, L., and Sánchez, J. (2016). Usability and UX evaluation of a mobile social application to increase students-faculty interactions. In Communications in Computer and Information Science (Vol. 618). & B2 \smallskip\\
		    \bottomrule 
		\end{tabular}%
	}{%
	\Fonte{Produced by the author.}%
%	\Nota{esta é uma nota, que diz que os dados são baseados na	regressão linear.}%
%	\Nota[Anotações]{uma anotação adicional, seguida de várias outras.}%
    }
    \end{table}

\section{Limitations}
\label{sec:conclusion-limitations}
Although the rigorous methodology used in this research achieved its main purposes and can contribute to the planning and guidance of cognitive impact evaluation, it has some limitations, called threats to the validity in \citeonline{Wohlin2000}. The use of Systematic Literature Review and Grounded Theory, methods well-regarded and commonly used in scientific research, reduce the limitation comparing to ad-hoc research. However, the use of each method has its limitations. Regarding the Systematic Literature Review, the procedures used in this study have deviated from Kitchenham’s guidelines \cite{Kitchenham2007} in several ways:
        \begin{itemize}
            \item The search was organized as a manual search process of a specific set of journals and conference proceedings not as an automated search process. 
            \item A single researcher selected the candidate studies, and also the studies included and excluded were checked by a single researcher. 
            % To reduce the bias, we apply a cross-validation selection to check the third filter of the systematic literature review (described in Section \ref{fig:cross_validation}).
            \item We consider not to have all existing studies about cognitive impact evaluation in the context of this study, which can imply in incomplete results. To mitigate it, we apply backward and forward snowballing.
        \end{itemize}

Concerning the Grounded Theory, we highlight some of the threats to validity such as \textit{(i)} the ones related with the interpretation bias and information abstraction during the coding process that directly affects the results of this work; and \textit{(ii)} concerned if the conclusions are reasonable and based on data. Then, the coding process in Grounded Theory could have had many biases, since only one researcher analyzed the data and extracted concepts. 

About the set of guidelines construction, no experimental studies to assess the validity of the proposed set were performed, neither a refinement with additional research data.

\section{Future Work}
\label{sec:conclusion-future}
After compiling the data from the systematic literature review and analyzing theoretical foundations, we conclude that there is a need to better plan and present data from experiments on technologies for people who are blind. With this, we aim to help improve the quality of the experiment itself and the interaction of the technology with respect to the cognitive objective. Faced with this nego, this study leaves the following future work:

    \begin{itemize}
        \item To expand the guidelines to more specific ones;
        \item To explore the visualization of the guidelines for the target public as an interactive website;
        \item To create a template offered in the initial stages of the research for the purpose of validation to improve the acceptance of the paper in the academic environment and to ensure its quality;
        \item To validate the guidelines with experts the guidelines by using an online form;
        \item To investigate the use of the guidelines in a real experiment of cognitive impact and evaluate the set of guidelines;
        \item To search for more information about the experiments from papers retrieved and improve the guidelines;
        \item To perform an evaluation of the experience of researchers in the experiment method and relate this and the quality of the experiment; and
        \item To expand and generalize the guidelines to comprise different cognitive processes less common, but more used in other types of application for people who are blind. 
    \end{itemize}

Nevertheless, as presented in this work, there are still challenges to be overcome. Throughout this research, we could observe that the research area of cognitive impact evaluation in the context of this work is recently presenting several possibilities for research to be conducted. Also, the cognitive impact field merges many subjects and have some different aspects from a current experiment. We hope that with the body of knowledge organized in this work and the possibilities of future work pointed out in this chapter, we could contribute to the progress in the fields of research.

%\section{Final Considerations}
%\label{sec:finalconsiderations}