\apendice{Papers retrieved from the Systematic Literature Review}
\label{ap:A}

This appendix presents the list of technologies retrieved in the papers from the Systematic Literature Review. The type of technology considered in this work is explained in the Section \ref{subsec:methodology-planning}. The Table \ref{tab:tools_evaluated} shows the list of technologies and the papers related.

\begin{small}

\begin{longtable}[h]{m{2.7cm}m{7cm}m{3cm}m{2cm}}
\captionsetup{width=16cm}%Deixe da mesma largura que a tabela
%\caption{Tools evaluated} \label{tab:tools_evaluated} \\
\caption{\label{tab:tools_evaluated} Technologies for people who are blind} \\
\toprule
Technology & Technology description & References & Type \\
\midrule \midrule

%\hline \multicolumn{1}{c}{Technology} & \multicolumn{1}{c}{Technology description} & \multicolumn{1}{c}{References} & \multicolumn{1}{c}{Type} \\ \hline 
\endfirsthead

\multicolumn{4}{l}%
{{ \tablename\ \thetable{} -- continued from previous page}} \\
%\hline \multicolumn{1}{c}{Technology} & \multicolumn{1}{c}{Technology description} & \multicolumn{1}{c}{References} & \multicolumn{1}{c}{Type} \\ \hline 
\toprule
Technology & Technology description & References & Type \\
\midrule \midrule

\endhead

\hline \multicolumn{4}{r}{{Continued on next page}} \\ \hline
\endfoot

\hline \hline
\endlastfoot
            AudioChile & A virtual environment that can be navigated through 3D sound to enhance spatially and immersion throughout the environment. & {\tiny \cite{Sanchez2005}} & Maps \\ \hline
            
            3D audio-based game & A navigational 3D audio-based game. The goal of the game consists in exploring a virtual environment while listening to 3D directional sounds that guide the player towards finding 5 hidden objects as quick as possible. & {\tiny \cite{Balan2014}} & Maps$/$Game \\ \hline
            
            AbES & An environment to be explores for the purposes of learning the layout of an unfamiliar, complex indoor environment. & {\tiny \cite{Connors2014,Connors2013,Sanchez2014,Jain2011}} & Maps \\ \hline
            
            ambientGPS & An audio-based software program embedded in a pocketPC that together with the assistance of a GPS satellite provides information about position, orientation and distance for blind learners’ navigation. & {\tiny \cite{Reads2009}} & Maps \\ \hline
            
            aMFS & Audible Mobility Feedback System. & {\tiny \cite{Adebiyi2017}} & Maps \\ \hline
            
            ATM machine & ATM usage fees are the fees that many banks and interbank networks charge for the use of their automated teller machines (ATMs). & {\tiny \cite{Shafiq2014}} & ATM \\ \hline
            
            Audio Haptic Maze (AHM) & AHM allows a school-age blind learner to be able to navigate through a series of mazes from a first-person perspective. & {\tiny \cite{ISI:000304018900033,Sanchez2013a,Sanchez2014Multimodal}} & Maps \\ \hline
            
            Audio-based GPS software & An audio-based GPS software program on navigation through open spaces. & {\tiny \cite{Sanchez2010autonomous}} & Maps \\ \hline
            
            AudioBattleShip & A sound-based interactive and collaborative environment for blind children. This system is a similar version of the traditional battleship game for sighted people but including both a graphical interface for sighted users and an audio-based interface for people who are blind. & {\tiny \cite{Sanchez2005b,Sanchez2004}} & Game \\ \hline
            
            AudioGene & A game that uses mobile and audio-based technology to assist the interaction between blind and sighted children and to learn genetic concepts. & {\tiny \cite{AudioGene}} & School learning$/$Game \\ \hline
            
            AudioLink & An interactive audio-based virtual environment for children with visual disabilities to support their learning of science. & {\tiny \cite{Sanchez2007science,Sanchez2009a}} & School learning \\ \hline
            
            AudioMath & An interactive virtual environment based on audio to develop and use short-term memory, and to assist mathematics learning of children. & {\tiny \cite{Sanchez2005c}} & School learning \\ \hline
            
            AudioMetro & An application software for blind users that represents a subway system in a desktop computer to assist mobilization and orientation in a subway network. & {\tiny \cite{Sanchez2006}} & Maps \\ \hline
            
            AudioMUD & A multiuser virtual environment for people who are blind. & {\tiny \cite{Sanchez2007a}} & Game \\ \hline
            
            AudioNature & An audio-based virtual simulator for science learning implemented in a mobile device (pocketPC) platform. & {\tiny \cite{Sanchez2008}} & School learning$/$Game \\ \hline
            
            Audiopolis & A videogame for navigating a virtual city through interaction with audio and haptic interfaces. & {\tiny \cite{Sanchez2014a,Sanchez2011,201353,Sanchez2014Multimodal}} & Maps$/$Game \\ \hline
            
            AudioStoryTeller & A tool for PocketPC to support the development of reading and writing skills in learners with visual disabilities (LWVD) through storytelling, providing diverse evaluation tools to measure those skills. & {\tiny \cite{Sancheza}} & School learning$/$Game \\ \hline
            
            AudioTransantiago & An audio-based software to assist people who are blind in public bus transportation. A handheld application that allows users to plan trips and provide contextual information during the journey using synthesized voices. & {\tiny \cite{Sanchez2013,Sanchez2011a}} & Maps \\ \hline
            
            BlindAid & The system allows the user to explore a virtual environment. & {\tiny \cite{201219,Schloerb2015,Lahav2012}} & Maps \\ \hline
            
            Building Navigator & A digital-map software with synthetic speech installed on a cell phone or PDA. & {\tiny \cite{20105}} & Maps \\ \hline
            
            Digital Clock Carpet (DCC) & An hour system for directions was used to tell the user how to get to the destination point. & {\tiny \cite{Sanchez2010b}} & Maps \\ \hline
            
            EyeCane & A hand-held device which instantaneously (50 Hz) transforms distance information via sound and vibration such that the closer an object is to the user the stronger the vibration of the haptic actuator and the higher the frequency of the auditory cues. & {\tiny \cite{Buchs2017}} & Maps \\ \hline
            
            HAGA & Haptic Audio Game Application (HAGA) is a software to assist in orientation and mobility (Maps) training by introducing blind users to a spatial layout. & {\tiny \cite{Merabet2016}} & Maps \\ \hline
            
            Indoor floor plans accessible & An automated approach that aids a visually impaired individual in obtaining information from a floor map, before visiting large buildings like a library. & {\tiny \cite{Paladugu2015}} & Maps \\ \hline
            
            Interactive audio map system & It enables blind and partially sighted users to explore and navigate city maps from the safety of their home using 3D audio and synthetic speech alone. & {\tiny \cite{Stojmenovic2014}} & Maps \\ \hline
            
            Métro & A software solution for blind users that represents a subway system. & {\tiny \cite{Sanchezb}} & Maps \\ \hline
            
            Métro Mobile (mBN) & A software solution for blind users that represents a subway system. & {\tiny \cite{Sanchezb}} & Maps \\ \hline
            
            Mobile devices & Mobile devices in general. & {\tiny \cite{Guerreiro2011}} & Mobile devices \\ \hline
            
            MOVA3D & The 3D video game MOVA3D uses 3D graphics and spatial sound which allows users to navigate freely through the virtual environment. & {\tiny \cite{Sanchez2010b}} & Maps$/$Game \\ \hline
            
            MovaWii & Based on these audio-haptic interface elements, consisting of the virtual representation of a real-life plaza through audio and haptic interfaces in which a learner who is blind has the goal of finding a lost jewel by using the Wiimote controllers. & {\tiny \cite{Sanchez2014Multimodal,SanchezVideogaming}} & Maps \\ \hline
            
            MVE & A haptic-based multi-sensory virtual environment enabling the exploration of unknown space. A multi-sensory virtual environment (MVE) with a haptic virtual environment enabling people who are blind to explore unknown spaces. & {\tiny \cite{Lahav2008Construction,Lahav2004}} & Maps \\ \hline
            
            MVLE & Multimodal-virtual-learning-environment (MVLE) is a virtual environment with audio and haptic feedback. & {\tiny \cite{Lahav2005}} & Maps \\ \hline
            
            NavTap & A navigational method that enables blind users to input text in a mobile device by reducing the associated cognitive load. & {\tiny \cite{Guerreiro2009}} & Input text \\ \hline
            
            Path-Guided Indoor Navigation & An omnipresent cellphone based active indoor wayfinding system for the visually impaired. & {\tiny \cite{Jain2014}} & Maps \\ \hline
            
            Reconfigured Mobile Android Phone (R-MAP) & The R-MAP is a fully integrated, stand-alone system that has an easy-to-use interface to reconfigure an Android mobile phone. & {\tiny \cite{Hossain2011}} & Mobile devices \\ \hline
            
            SiFASo & SiFASo (Simulative environment for acoustic 3D software) is an assistive technology for sightless. Persons based on a virtual auditory display (VAD) intended to encourage its users to improve their map-forming skills and use them more efficiently while providing a safe virtual sound environment to virtually walk and navigate through. & {\tiny \cite{Ohuchi2006}} & Maps \\ \hline
            
            TactiPad & An academic prototype of a tangible multimodal interface that has an interface similar to a Braille matrix. & {\tiny \cite{Pissaloux2018a}} & Device \\ \hline
            
            Theo \& Seth & An audio-based virtual environment to enhance the learning of mathematics knowledge in blind children. It is a game-based virtual environment that includes interesting mathematics learning activities with different levels of complexity. & {\tiny \cite{Sanchez2005}} & School learning \\ \hline
            
            Timbremap & A sonification interface enabling visually impaired users to explore complex indoor layouts using off-the-shelf touch-screen mobile devices. & {\tiny \cite{Su2010}} & Maps \\ \hline
            
            Tower Defense & A video game that allows users who are blind to gradually build up a mental model based on references between different points on a Cartesian plane, in a way that is both didactic and entertaining. & {\tiny \cite{Espinoza2014}} & School learning$/$Game \\ \hline
            
            UnrealEd & A 3D virtual environment & {\tiny \cite{Villane2009}} & Maps \\ \hline
            
            Virtual environment & A virtual environment (VE) that provides haptic and audio feedback to explore an unknown space. An aural environment with navigable structures with only spatial sound. & {\tiny \cite{Lahav2008b,Sanchez2000}} & Maps \\ \hline
            
            VirtualLeap and VirtualWalk & VirtualLeap, which allows the user to jump through a sequence of street intersection labels, turn-by-turn instructions and POIs along the route; VirtualWalk, which simulates variable speed step-by-step walking using audio effects, while conveying similar route information. & {\tiny \cite{Guerreiro2017}} & Maps \\ \hline
            
            vMFS & Vibrotactile mobility feedback system & {\tiny \cite{Adebiyi2017}} & Maps \\ \hline
            
            Wolfpack Haptic Virtual Environment & Using the Direct X SDK and Novint HDAL SDK, a haptically enhanced software that allows a user to create various objects, such as a sphere, cube, and cylinder, in a 3D virtual environment. & {\tiny \cite{Lee2014a}} & Objects structure learning \\ 

            
\end{longtable}

\end{small}