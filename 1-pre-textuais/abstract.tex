Visual disability has a major impact on people’s quality of life. Although there are many technologies to assist people who are blind, most of them do not necessarily guarantee the effectiveness of the intended use. As part of research developed at the University of Chile since 1994, we investigate the interfaces for people who are blind regarding a gap in cognitive impact, which include a broad spectrum of the cognitive process. In this work, first, a systematic literature review concerning the cognitive impact evaluation of multimodal interface for people who are blind was conducted. The study selection criteria include the papers which present technology with a multimodal interfaces for people who are blind that use a method to evaluate the cognitive impact of interfaces. Then, the results of the systematic literature review were reported with the purpose of understanding how the cognitive impact is currently evaluated when using multimodal interfaces for people who are blind. Among forty-seven papers retrieved from the systematic review, a high diversity of experiments was found. Some of them do not present the data results clearly and do not apply a statistical method to guarantee the results. Besides this, other points related to the experiments were analyzed. The conclusion was there is a need to better plan and present data from experiments on technologies for cognition of people who are blind. Moreover, this work also presented a data qualitative analysis based on the Grounded Theory based method to complement and enrich the systematic review results. Finally, a set of guidelines to conduct experiments concerning the cognitive impact evaluation of multimodal interfaces for people who are blind are presented.


% Separe as Keywords por ponto
\keywords{Impact Evaluation. Cognitive Evaluation. Multimodal Interfaces. People who are Blind.}