This master's degree was a unique opportunity with moments of exceptional academic and personal learning. The end of this stage is felt with gratitude to those who collaborated and the organizations that made it possible.

First, I am very grateful to Professor Jaime Sánchez for guiding me and for trusting me. Thank you for having applied so much energy in the papers and in this thesis dissertation. I appreciate having had the opportunity to participate in your research on Human-Computer Interaction focused on interfaces for people who are blind, which has been developed for more than 20 years with great commitment and dedication. Professor Jaime Sánchez, thank you very much for all your advice and for having encouraged the completion of this cycle.

I thank Professor Antônio Macedo who has always attended to the requests and always responded quickly and simply. I thank Professor Rossana Andrade for starting this process and making it possible.

I extend thanks to the members of the committee for the availability and contributions. Thanks to Professor Fran Oliveira for the excellent contributions. I consider you an advisor in my professional life and I admire your sensitivity to the Human-Computer Interaction area. Thanks to Professor Windson Vieira for the contributions that have been made since the first steps of the research. Thank you for encouraging me to return to the academia.

I thank the professors who inspired me, Professor Reinaldo Braga, to whom I have great admiration, and Professor José Marques, an advisor who made me have a high-quality parameter in this role.

During the research, I had incredible support from the people around me, who had to understand my absence and shared great moments of achievement.

I would like to thank all my friends and highlight some who have been around this time. I am very grateful to the friends who shared lessons, presentations, coffees, studies, and rides. Among them are Deborah Magalhães, Mariana Carneiro, Rainara Carvalho, Italo Linhares, Josy Vale, Gabriel Neemias, Tales Paiva, Paulo Artur, and Rodrigo Almeida. Special thanks to Jefferson Barbosa, who divided the master's daily and followed the process closely.

My sincere thanks to my friends Andrea Oliveira and Suelen Moura for their friendship, affection, understanding and philosophical teaching. Thanks to my friends Felipe Barbosa, for the understanding, Cris Mayara, for the encouragement, Vladymir Bezerra, for the inspiration, Thayanne Albuquerque, for the smiles, Cecília Costa, for your support, Muriel Brunello, for your sweet affection, and Lívia Rodrigues, for being a safe haven. Thanks also to Agebson Façanha, with whom I shared knowledge in the field, and to Ticianne Darin who contributed to the research.

I would like to thank all the teachers and staff of MDCC and GREat, especially Janaína, Darilu and Hudson, Gláucia Mota and Jonatas Martins. Thank you for the benefits that have been granted to me at this time in the university.

I am very grateful to my family for the care and support, especially for my cousins Clécia Klysmann, for the friendship, words, care and support, and  Leiri Mesquita, for the invaluable presence, and for my godfathers Zanilton Medeiros, Salomé Mesquista and Carlos Abreu who accompanied me as parents. I am incredibly grateful to my grandmother, Antônia Batista, who taught me during the master's degree the strength of the study, telling me the size of her effort to learn to read in a rural town in Ceará. She was one of those who had to be more understandable for my lack, but she never failed to understand the dedication I needed to finish this master degree. I always had support and inspiration from her.

I thank my parents, Herculano Mesquita and Zélia Medeiros, and my partner André Fontes, who gave me support in the emotional, financial, ethical and even technical aspects. When I decided to pursue the master degree, they were willing to help me on this journey and they fulfilled it. The difficulties and joys of this cycle have become ours.

Finally, I thank \textit{Fundação Cearense de Apoio ao Desenvolvimento} (Funcap), in the person of President Tarcisio Haroldo Cavalcante Pequeno, for the financing of the master research through a scholarship. Also, the development of this Master Thesis is part of the Research Project of the National Fund for Science and Technology, Fondecyt-Chile \# 1150898,``Knowing with Multimodal Interfaces in Blind Learners'', as well as being part of the Basal Funds for Centers of Excellence, FB0003, CONICYT-Chile.



