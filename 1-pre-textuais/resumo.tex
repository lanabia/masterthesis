A deficiência visual tem um grande impacto na qualidade de vida das pessoas. Embora existam muitas tecnologias para ajudar as pessoas cegas, a maioria delas não garante necessariamente a eficácia do uso pretendido. Como parte da pesquisa desenvolvida na Universidade do Chile desde 1994, investigamos as interfaces para pessoas cegas quanto ao impacto cognitivo, que inclui um amplo espectro do processo cognitivo. Neste trabalho, primeiramente, realizou-se uma Revisão Sistemática da literatura sobre a avaliação do impacto cognitivo de interfaces multimodais para pessoas cegas. Os critérios de seleção do estudo incluem artigos que apresentam tecnologias com interfaces multimodais para pessoas cegas e que usam algum método para avaliar o impacto cognitivo destas interfaces. Em seguida, os resultados da Revisão Sistemática da literatura foram relatados com o objetivo de compreender como o impacto cognitivo é avaliado atualmente quando se utiliza interfaces multimodais para pessoas que são cegas. Entre os quarenta e sete artigos recuperados da Revisão Sistemática, uma alta diversidade de experimentos foi encontrada. Alguns deles não apresentam os dados de resultados de forma clara e não aplicam um método estatístico para garantir as conclusões. Além disso, outros pontos relacionados aos experimentos foram analisados. Nota-se como conclusão que há uma necessidade de planejar melhor e apresentar os dados dos experimentos em tecnologias que focam no aprimoramento cognitivo de pessoas cegas. Além disso, este trabalho também apresentou uma análise qualitativa de dados baseada no método  de Teoria Fundamentada nos Dados \textit{(Grounded Theory)} para complementar e enriquecer os resultados da Revisão Sistemática. Finalmente, apresenta-se um conjunto de diretrizes para realizar experimentos de avaliação do impacto cognitivo em interfaces multimodais para pessoas que são cegas.

% Separe as palavras-chave por ponto
\palavraschave{Avaliação de Impacto. Avaliação Cognitiva. Interfaces Multimodais. Pessoas Cegas.}